\chapter*{总结与展望\markboth{总结与展望}{}}
\addcontentsline{toc}{chapter}{总结与展望}
本文借助符号计算工具 Mathematica 对变系数 Sasa-Satsuma 方程和 6 阶 KdV 方程进行了解析研究并开发了相应的简化计算的程序包。
论文的主要工作有以下方面:

1. 变系数 Sasa-Satsuma 方程。本章对变系数 Sasa-Satsuma 方程进行了解析研究、怪波求解、怪波模拟与分析。首先,对 Sasa-Satsuma 方程的研究背景及现状进行了阐述,通过 Painlev\'{e} 检测得到了方程的两组 Painlev\'{e} 可积条件,选择其中一组可积条件,根据变系数方程的形式扩展了 AKNS 形式求得了 3$\times$ 3 阶 Lax 对,基于得到的 Lax 对求得了 $\Gamma$-Riccati 形式的 B\"{a}cklund 变换,进而通过种子解得到了变系数 Sasa-Satsuma 方程的单孤子解;另外通过 Lax 对推导了变系数 Sasa-Satsuma 方程的无穷守恒律;最后用 Darboux 变换方法求到了变系数 Sasa-Satsuma 方程的怪波解,并用 Mathematica 对怪波解进行了模拟与分析,详细阐明了变系数 Sasa-Satsuma 方程的各个变系数对方程怪波的波形及传播产生的影响。

2. 6 阶 KdV 方程的解析研究。本章对 6 阶 KdV 方程进行了解析研究。首先,通过贝尔多项式方法推导出 6 阶 KdV 方程的双线性形式;基于贝尔多项式形式的双线性形式求得了方程的 N 孤子解、B\"{a}cklund 变换和 Lax 对,最后通过 Lax 对推导出方程的无穷守恒律。

3. 怪波求解程序包开发。
本章主要开发了求解方程怪波解的程序包。借助符号计算工具 Mathematica 开发一个可以自动求解方程怪波解的程序包,该程序包采用的算法不是直接求解方程怪波解,而是通过変量変换法寻找两个方程的关系,再将得到的关系式应用于其中一个方程的已知怪波解就可以得到另一个方程的怪波解。封装程序包减少了手工计算的复杂度,保证了计算的正确性并提高了效率。

目前对变系数 Sasa-Satsuma 方程和 6 阶 KdV 方程的研究和程序包的开发都取得了较好的结果,但是仍然有许多方面需要进一步研究,可以作为下一步研究的重点:

1. 本文通过 Darboux 变换方法求解了变系数 Sasa-Satsuma 方程的一阶怪波对,接下来可以尝试其它方法对变系数 Sasa-Satsuma 方程的高阶怪波进行研究。

2. 本文对 6 阶 KdV 方程进行了解析研究,但对该方程的怪波研究未取得实质性进展,目前对怪波的研究更多地应用于薛定谔方程,KdV 方程的怪波解研究将成为下一步研究的重点。

3. 本文开发的程序包通过変量変换方法间接求出方程的怪波解,如果对怪波解的研究更加深入,获得规律,可以开发直接求解方程怪波的程序包。

