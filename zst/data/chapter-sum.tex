\chapter*{总结与展望\markboth{总结与展望}{}}
\addcontentsline{toc}{chapter}{总结与展望}
本论文借助符号计算工具 Mathematica 对变系数的 Sasa-Satsuma 方程和空间变量非局域的变系数 Sasa-Satsuma 方程进行了解析研究并开发了用于求解方程守恒律的程序包。
论文的主要工作有以下几个方面:

1. 对变系数的 Sasa-Satsuma 方程进行了解析研究。首先对变系数的 Sasa-Satsuma 方程背景进行简要的介绍,并与之前已经研究过的类似的方程进行对比和分析,通过 Painlev\'{e} 分析得到了方程的可积条件,在此基础上通过扩展的 AKNS 系统构造了方程的 $3 \times 3$ 阶的 Lax 对,然后基于得到的 Lax 对获得了原方程的 Riccati 形式的自 B\"{a}cklund变换,并使用种子解和该变换得到了原方程的单孤子解,另外又通过 Lax 对推导了方程无穷守恒律的关系式,并列举给出了前三组的守恒律,最后借助得到的孤子解,对孤立波进行模拟并分析了原方程系数对孤立波的波形和传播过程产生的影响。

2. 对非局域的变系数 Sasa-Satsuma 方程进行了解析研究。这部分的内容是在前面的基础上进行研究的。所研究的目标方程是一个含有 7 个关于 $t$ 函数的变系数非局域非线性薛定谔方程。研究的主要内容和前面的内容基本类似。首先对非局域的相关方程的研究背景进行了介绍,然后通过扩展的 AKNS 系统构造了方程的 Lax 对,然后通过得到的 Lax 对得到原方程的 B\"{a}cklund变换,并在此基础上通过零解得到了原方程的单孤子解,此外还推导得出了方程的无穷守恒律,最后使用 Mathematica 对方程的孤子解进行模拟。

3. 守恒律求解程序包开发。借助符号计算工具 Mathematica,对方程守恒律求解过程进行了形式化, 开发一个可以自动求解前 $n$ 项守恒律的程序包。该程序包通过输入方程守恒律的迭代关系式和必要系数的通项公式以及起始项便可自动求解出方程前几组的守恒律,不仅减少了手工计算的复杂性,而且还保证计算的正确性并且提高了研究效率。

论文对目标方程的解析研究和程序包的开发都达到一个较好的结果,但是仍然有一些不足的地方,有许多方面还需要进一步的研究,并且可以作为下一步研究的重点:

1. 本论文通过  B\"{a}cklund 变换的形式获得了变系数的Sasa-Satsuma 方程的单孤子解,可以进一步考虑求得多孤子解,此外还可以尝试其它的方式求得方程其它形式的解,比如怪波解。

2. 本论文对非局域的变系数 Sasa-Satsuma 方程进行了研究,但是该方程只包含了空间变量的局域性,未包含时间变量的局域性,因此可以考虑加入时间变量局域性的条件后进行研究,此外还可以进一步研究方程其它的性质如汉密尔顿结构,怪波解等。

3. 本论文开发的程序包只能对所研究方程包含的范围之内以及与其相类似方程求出守恒律,但是对于其它形式或者更为复杂的方程就不会得到正确的结果,因此还可以对其进行改进,使其适用性更为广泛。