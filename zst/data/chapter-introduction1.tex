% !Mode:: "TeX:UTF-8"
\chapter{绪论}
\section{研究背景}
\subsection{符号计算}
计算机科学及其相关技术的飞速发展正改变着我们的工作和生活方式,从单纯的计算到文字处理,再到知识表示与图形变换,计算机已经成为科学研究和工程领域不可缺少的工具。计算机相关技术的应用也已经渗透到各行各业并且发挥着越来越重要的作用。这也使得计算机科学与其它学科关系越来越密切,并且融合产生了新的学科和分支。例如在数学相关领域,随着计算机技术的增强,使得人们对能够对特殊问题通过计算求解而非只关注于问题本身的数学结构。迄今为止,计算机科学已经对数学、流体力学以及工程等领域产生了重要影响\upcite{syc-1,syc-5}。

在这种情况下,通过计算机进行对数学表示式的处理以及数学算法的研究已经形成了一门新的学科分支——计算机代数,也被称之为符号计算。它是区别于数值计算的另一种计算形式。计算机代数,顾名思义,主要是研究如何使用计算机进行数学公式的推导与运算,比如求解微分方程或者积分方程,矩阵运算,函数的微分与积分运算等。它研究对象的主体主要是数学符号、表达式和其它数学对象。数值计算和符号计算都是计算机科学计算的两种方式,然而数值计算过程是变量值、函数值、数值之间的相互变换,这些值只能为某一精度范围内的数值,无法应用于对精度要求很高的某些数学领域或者是现实世界中的某些实际问题场景。而符号计算能够达到数学上完全无误差的精确运算,因此与数值计算相比,符号计算对于计算机硬件和软件的要求更高\upcite{syc-1}。此外数值计算和符号计算的工具和原理同样都依赖很多数学上的算法。但是对于这些数学算法而言,符号计算相比数值计算能继承更多且更丰富的数学遗产。无论是古典时期的数学算法还是近现代数学中的算法都在符号计算中广泛的使用并是符号计算系统的核心算法成员。而近些年成熟的数学算法的研究成果也源源不断的充实到符号计算中\upcite{so-4}。

能够执行符号计算的应用程序软件被称为符号计算系统,同样也被称为计算机代数系统。它是一个高度集成化的可用来表示数学知识的数学工具系统。通常,一个符号计算系统也同时包含数值计算的功能,此外还会包含程序设计语言和图形结果演示这几个部分。这些功能相互融合使得数值计算、公式推导、表达式运算和图形可视化操作保持了一致性和连贯性。一般符号计算系统会内置大多数数学领域的算法,从初等数学到高等数学,几乎会涉及所有的数学分支学科,其中包括数学函数、矩阵和线性代数、公式操作、方程求解、微积分、最优化、概率和统计、离散数学、几何学、布尔计算等。符号计算已经成功的应用于几乎所有的科学技术和工程领域,其中就包括了非线性孤子领域。符号计算的出现和发展在一定程度上改变了数学等领域传统的研究手段,并且也在相关领域例如孤子理论的解析研究中的得到了长足的发展\upcite{so-2, syc-12}。

科学研究中常常需要不同的研究手段和方式,有时可能是实验性的,而有时又可能是理论性的。例如有些科研中需要进行大量的公式推导和验证,有时由于模型的设计或者演算过程过于复杂,人工计算往往不可靠甚至是不可行,这时就不得不借助计算机符号计算系统的帮助。例如 19 世纪时期的法国天文学家 Charles Delaunay 把月亮的位置视为时间的函数,从 1847 年到 1867 年经过了 20 年的时间推导了近四万个数学公式,完成了长达数百页关于计算方面的文章。然而到了 1970 年 MIT 的一个以 Drprit 为首的研究小组用符号计算软件只花了 20 小时的时间便完成对 Delaunay 计算公式的复算,复算结果表明原先的计算存在 3 个错误\upcite{syc-5,syc-11}。这一个很有代表性的例子,表明符号计算系统软件确实能提高研究结果的正确性以及研究效率。

目前符号计算系统软件已有几十种,具体可分为专用系统软件和通用系统软件两类。通用符号计算系统一般都会具有数值计算、符号计算、程序设计语言和图形可视化操作等功能。常用的通用符号计算系统有: Derive、Reduce、Macsyma、Maple、MuMath、Mathematica 等\upcite{syc-3,syc-5}。符号计算系统通常有两种交互操作方式:一种是输入命令就会执行一种相应的数学计算;另一种就是写一段程序,交由符号计算系统解释执行,这种方式更为灵活。一般符号计算系统都会有自己的编程语言,这些语言通常和通用的高级程序语言如 C 研究都很类似\upcite{syc-1}。

在这些符号计算软件中,最广为人知并被广泛使用的当属 Wolfram Mathematica。Mathematica 是由科学家 Wolfram 创办公司开发的符号计算软件。从 1988 年开发出的 1.0 版本,经过几十年不断迭代扩充和修改后,到 2018 年 3 月为止,Mathematica 已经发布 11.3 的稳定版本。它拥有强大的数值计算和符号运算能力,在这几十年来被广泛使用于科学、工程、数学、计算等领域。用户可以使用 Mathematica 的内置语言 Wolfram 编写程序使用其功能,也可以使用 Package 的形式开发并发布自己的程序包。目前,Mathematica 的功能不仅仅局限于数学学科领域的使用,其它的学科领域可能充分借助其强大的功能。这些领域包括图像处理、声音信号处理、地理数据计算、气候数据计算、生命科学和医学数据计算、物理化学数据计算、天文和地球科学数据计算、金融数据计算、社会文化与语言经济数据计算和工程数据计算等。更详细的内容可以参考 Mathematica 官方文档指南$\footnote{https://reference.wolfram.com/language}$。


\subsection{非线性科学}
非线性科学是一门从 20 世纪 60 年代以来,在各门以非线性因素为特征的分支学科的基础上发展起来的,用于研究现实世界中非线性现象共性的综合性的交叉学科。它几乎涉及到了科学和工程的各个领域,并且正在重新改变人们对现实世界的看法\upcite{syc-12,so-1,so-2}。在许多自然科学、社会科学的研究和工程实践中,并不是所有的问题都能使用线性模型就能解决,这就使得非线性相关模型的研究就显得十分的重要。非线性偏微分方程作为数学模型可以用来描述出现在物理学、信息科学等领域的现象,而方程的解和解的性质可以用来揭示隐藏在这些现象背后的原理和原因,因此非线性偏微分方程和其解析研究就成为了非线性科学中一个很重要的组成部分。实际中的非线性问题往往十分的复杂且多样,能从中得到数学模型方程十分的繁多,而其中每一类的数学模型方程都可以解释实际中的一类或者几类的问题。例如从水波的研究可以得到传统的 KdV 方程,对生物种群的变化生态模式研究可以得到 Logistic 方程,描述湍流发生的 Landau 方程等\upcite{so-3}。除此之外,常见的经典非线性方程还有非线性薛定谔方程、非线性光学方程、非线性热传导方程、非线性电报方程、色散耗方程等\upcite{so-2,so-3}。非线性科学研究中一个很困难的地方是目前尚无一个统一有效的方法去获得方程的解及相关性质,往往一种方法只能用于其中的一类或者是几类的方程,这也是为什么在非线性方程的研究中,很多文献都关注用过去已有的方法修改或扩展使其能够求解新的方程,亦或是提出一种全新的方法来去确定方程的解,同时这也说明非线性科学还有大量可探索的空间。

非线性科学研究对象的主体可分为以下几个方面:混沌(Chaos)、分形(Fractral)和孤子(Soliton)。而本论文的研究只涉及到孤子方面。孤子的概念是指具有弹性散射性质的孤立波,在 20 世纪 50 年代被提出且迅速成为非线性科学中的研究热点。孤子理论也成为物理、计算机科学和应用数学相交叉的一门学科,其涉及到的非线性方程也在工程领域有着广泛的应用。在数学上,孤子可以被解释为非线性偏微分方程的局部波动解。由于这些方程中的非线性项和色散效应达到巧妙的平衡,使得其能量始终集中在狭小的空间范围内,从而使孤立波在传播的过程中保持形状的不变。 而孤立波最早在 1834 年由英国科学家、造船工程师 John Scott Russel 发现。后随着研究的发展和深入,很多领域都发现了孤立波的现象,其中最典型的当属海洋中的内孤立波现象。海洋内孤立波是在海洋中发现的一种稳定的波动现象,其形成原因是不同深度海水吸收太阳辐射能不同导致海水的密度不同,在一些偶发因素的驱使下而形成的一种孤波现象。因此对孤立子的研究也将有助于对海洋科学的研究\upcite{syc-17,syc-18,so-2}。

目前,符号计算已经在非线性科学的研究中发挥着非常重要的作用。符号计算系统的进行推导时的高精确的特点非常适合非线性偏微分方程相关的研究。随着非线性相关理论的深入研究以及符号计算技术的快速发展,基于符号计算的非线性偏微分方程的解析研究已经成为目前非线性科学中最重要的研究分支之一。

\section{研究现状}
近年来以非线性偏微分方程为主要研究对象的孤立子理论发展十分迅速,形成了许多有效的研究方法,例如反散射法 (IST)、B\"{a}cklund 变换法、Painlev\'{e} 有限展开法、Lie 群法、Hirota 双线性法和 Darboux 变换法等。伴随着各式各样解析方法的出现,许多新的方程和新的解都不断的被发现和利用。下面就关于孤立子的解析方法的发展做一些简单的介绍。

1967 年, Gardrier 等人运用逆散射法求解了 KdV 方程并获得了成功,该方法最早起源于量子力学中,人们通过寻找正散射我呢提和反散射问题之间的关系,也就是 Schr\"{o}dinger 方程的特征值问题及其反问题之间的关系,即把方程通过变换转为线性可积的方程来去求得 KdV 方程初值问题的解。此后,人们又用这种方法成功求解了其它的一些非线性偏微分方程。这使得逆散射法逐步发展成为一种成功的数学物理方法。反散射方法的出现为数学,特别是非线性领域相关的研究提供了新的思路,并且对其产生了极为深远的影响。而后到了 1968 年,Lax 把应用于 KdV 方程的反散射法加以推广并理论化,给出了一个比较普遍的格式,使之可以方便的应用于更多非线性偏微分的方程。1972 年,Zakharov 和 Shabat 将 Lax 的思想用于求解 Schr\"{o}dinger 方程,并给出其精确的解析解。除此之外,Ablowitz, Kaup, Newell 和 Segur 等人也为泛化反散射法做出了杰出的贡献。1975 年,Wahl quit 和 Estabrook 借助于微分形式给出了延拓结构法,该方法可以寻找与反散射法相联系的线性特征值问题\upcite{so-2}。

1971 年,Hirota 提出了一种新的简单而又直接的方法可以用来获得非线性偏微分方程的孤立子解。在这个方法中,通过引入两个函数的双线性导数的概念,使得目标方程可以转化为双线性的导数方程。然后将扰动展开式代入到双线性方程中,就可以得到目标方程的单孤子解、双孤子解甚至是 $N$ 孤子解。随后该方法迅速的在非线性偏微分方程的研究中流行起来,直到今日仍然被广泛的使用\upcite{so-2}。

1989 年,兰慧彬等人提出的双曲正切函数展开法,使得以其成为代表的函数展开法成为求解非线性偏微分方程的一个重要技巧。1992 年,Malfiet 等人将这种方法系统化为构造非线性偏微分方程孤立波解的 tanh 函数法。该方法的基本思想是将方程的孤子解假设为双曲正切函数的多项式形式,从而将非线性偏微分方程的孤子求解问题转化为了普通代数方程组的问题。90 年代中期,Parks 和李志斌等人借助符号计算使用计算机进行了大量的计算,从而求解出了更为复杂的非线性偏微分方程\upcite{so-2}。

Jacobi 椭圆函数展开法作为另外一种函数展开法也是一种常见用于求解非线性偏微分方程的方法。该方法是由刘式适等人提出并用于解决某些方程的冲击波解和孤立波解等问题。随后陈怀堂等人该方法进行了各种扩展和推广,使其能够求得很多类型方程的精确解。到 2003 年,王明亮、周宇斌等人基于此方法进一步研究并提出了 F-函数展开法\upcite{so-2,so-1,so-3}。

齐次平衡法,同时也被称为拟解法,是由王明亮、李志斌等人提出的用于构造非线性偏微分方程孤立波解的一种有效的方法。通过使用该方法,可以预先的判定目标方程是否存在一定形式的精确解。这对于后续方程的求解非常重要。如果判断为存在,则可以通过一定的步骤求解出来。因此,齐次平衡法具有直接简洁、步骤分明等优良的特点\upcite{so-1,so-2,so-3,so-4}。该方法的具体原理同 Cole-Hopf 变换有些类似,是从非线性偏微分方程的结构出发,通过分析其非线性的特点、色散和耗散的阶数等因素的方式,并根据它们的最高阶数可以部分平衡的原则,从而确定其中某些方程解的一般形式。然后再将该形式的解代入到原目标方程中使其平衡,得到待定的方程组,最后求解这些方程组就可以得原方程的解\upcite{so-1,so-2}。

以上就是对非线性偏微分方程常见的解析方法做了简要的介绍,但是随着非线性研究领域的发展和深入,人们面对方程的形式以及计算求解过程变得越来越复杂,纯粹的人工计算基本上已经不可能存在,解析研究也越来越依托于符号计算系统的发展。在这种情况下,许多基于符号计算软件用于自动求解方程或者用于研究方程各种性质的程序包被开发出来。在方程的可积性方面,既有用于检测偏微分方程 Painlev\'{e}可积性的 Maple 程序包“SPIC”,也有用于检测常微分方程的  Macsyma 程序包“ODEPAINLEV\'{e}”。2006年,Hereman,Baldwin 等人在此基础上进一步开发  Mathematica 程序包“Painlev\'{e}Test.m”,它可以适用于多项式形式的微分方程 Painlev\'{e} 分析,而且直至今日仍被广泛的使用。但是目前用于判定偏微分方程诸如 B\"{a}cklund 变换,Lax 可积性等性质的程序包还没有出现\upcite{syc-3,syc-8}。

在方程求解方面同样存在许多具有使用价值的程序。1982年,Sehwarz 等人在 Reduce 上开发了对微分方程古典对称群求解的程序包。1989 年,Reid 和 Mausfield 等人分别开发出了用于简化微分方程组和用于自动化计算微分方程对称群的  Maple 程序包。1994年,李志斌开发出了能够高效求出许多经典非线性偏微分方程的精确孤立子解的程序包,并且在此基础上,他与柳银萍等人在非线性偏微分方程的自动求解上开展了各种工作,开发出了一系列的程序包。2008年,Liang 等人开发出基于扩展的 Tanh 方法的 Maple 程序包 TWS。但是由于非线性偏微分方程以及其研究方法的多样性,以上程序包大多只能用于某些方程特殊解的特定解法,开发适用性更广的用于方程求解的程序包有助于非线性研究领域的进一步发展\upcite{syc-3,syc-5,syc-8}。

\section{主要研究内容}

本论文的研究内容主要是利用符号计算工具 Mathematica 对两类方程进行了研究,其中第一类是变系数的 Sasa-Satsuma 方程,而第二类是空间局域的变系数 Sasa-Satsuma 方程。对方程的解析研究内容主要包括  Painlev\'{e} 可积性质分析、Lax 对、B\"{a}cklund 变换、孤子解、无穷守恒律以及孤子解传播的模拟等。进而在对方程守恒律等性质研究方法充分了解的基础上,编写基于 Mathematica 程序包对守恒律的求解进行自动化,减轻了人工计算的难度,同时在一定程度上践行了数学机械化的思想。

本文主要进行以下几个方面的研究:

第一部分为变系数的 Sasa-Satsuma 方程的解析研究及孤立波传播模拟。本部分所研究的  Sasa-Satsuma 方程同时包含 6 项关于 $t$ 函数的变系数,这使得在求解方面会变得有些复杂,但是求得解范围也更广,也能描述更多的非线性现象。首先,本部分研究方程的 Painlev\'{e} 可积性质,并在其中一组相容条件下扩展了 AKNS系统求得了方程的 Lax 对。在所求得的 Lax 对的基础上,得到了方程的自 B\"{a}cklund变换,进而根据该变换得到了方程的单孤子解。此外还引入Riccati 辅助变量推导了方程的无穷守恒律,列出并且验证了前几组的守恒律。最后借助之前所求得的单孤子解,进行了孤立波的模拟并分析了方程各项的系数对于孤立波传播的影响。

第二部分为局域变系数的 Sasa-Satsuma 方程的解析研究。本部分是在之前研究的基础进行更加深入的探索,使用的方法以及研究的性质和之前的基本相同,但是由于增加了空间局域的条件,求解方面就变得更加的复杂以及更有难度。本部分所研究的方程包含 7 项 关于 $t$ 函数的变系数,并且空间变量增加了局域性的条件,这使得方程求解与之前的研究相比变得更加复杂。本部分首先构造并求得了方程的 Lax 对,再通过 Lax 对得到了方程的自 B\"{a}cklund变换,然后结合种子零解得到了方程的单孤子解,并且研究了方程的无穷守恒律。

最后一部分为自动求解方程守恒律程序包的开发。本部分是在前面两部分研究的基础上,借助符号计算工具 Mathematica 开发一个可以自动获得方程守恒律的程序包。该程序包通过输入方程守恒律的迭代关系式和必要系数的通项公式以及起始项,自动求解出方程前几组的守恒律,不仅减轻了手工计算的难度,而且还能保证结果的正确性,提高了研究计算效率。

\section{研究意义}
随着计算机科学及其相关技术的快速发展,符号计算系统越来越强大,如何将符号计算工具应用于更多的科学工程领域并帮助其发展变得更有意义。作为数学、物理、计算机等众多交叉学科都涉及到的非线性相关问题一直是学术界的研究的重点和难点,尤其是对用来描述现实世界中诸多非线性现象的非线性偏微分方程研究更是具有重要的意义。

非线性偏微分方程的研究最大的困难之一在于解析解的获得十分困难,不像线性的常微分方程具有相较统一的求解方式,不同的非线性偏微分方程一般有着不同的求解方法,而这些方法又往往比较复杂。近年来,大量的经典书籍和科研文献都致力于将已有方法推广到更多的非线性偏微分方程或者是获得更多新的用于求解的方法。因此,解析研究将直接影响非线性偏微分方程的理论研究和发展。

如何更好的利用符号计算系统工具 Mathematica 来对相关的非线性偏微分方程的可积性以及解析解等性质进行研究是本文研究的重点。随着对非线性偏微分方程深入研究,所研究的方程以及其计算过程变得越来越复杂,人工计算变得越来越不可靠,因而必须借助相关的符号计算系统。利用 Mathematica 对非线性偏微分方程的解析性质进行研究,对孤立波进行模拟,并将方程的某些性质求解进行符号化,编写出可靠实用的 Mathematica 程序包是本文的主要研究内容。此外本文的研究成果对某些非线性偏微分方程的研究会起到一定的促进作用。

\section{论文结构安排}
本论文总共包含五个章节,其中的第一、二章节主要介绍论文的研究背景和所需要的基本的数学理论和方法;第三、四、五章节是论文的主要研究内容。论文详细的安排如下:

第一章,绪论。本章首先对作为论文研究重点的符号计算和非线性偏微分方程的背景知识进行了简单的介绍,进而对孤子研究领域和符号演算的研究发展进行了总结和概括,然后阐述了本论文的主要研究内容和意义,最后给出了论文内容的结构安排。

第二章,相关的数学理论和方法。本章对之后几个章节的研究中所用到的基本的数学方法和理论进行了概述,其中所使用的数学方法包括 Painlev\'{e} 可积性质的分析、Lax可积、B\"{a}cklund变换、无穷守恒律和孤子解的求解。这些内容都是后续对非线性偏微分方程的研究中所必须使用到的。

第三章,变系数的 Sasa-Satsuma 方程。本章节对一个含有 6 个关于 $t$ 函数的变系数非线性薛定谔方程进行了研究。研究内容主要包括方程的 Painlev\'{e} 可积性质、Lax 对、单孤子解、无穷守恒律和孤立波的模拟。详细的内容如下:首先对于变系数的 Sasa-Satsuma 方程相关的研究背景进行了阐述,进而通过  Painlev\'{e} 检测得到了方程的  Painlev\'{e} 可积条件,在此基础上通过扩展的 AKNS 系统构造了方程的 $3 \times 3$ 阶的 Lax 对,然后基于得到的 Lax 对获得了原方程的 Riccati 形式的自 B\"{a}cklund变换,并使用种子解和该变换得到了原方程的单孤子解,另外又通过 Lax 对推导了方程无穷守恒律的关系式,并列举给出了前三组的守恒律,最后借助得到的孤子解,对孤立波进行模拟并分析了原方程系数对孤立波的波形和传播过程产生的影响。

第四章,变系数的非局域的 Sasa-Satsuma 方程。本章节是在前一章的基础上进行研究的,也可以说是前一章研究内容的延续。本章节对一个含有 7 个关于 $t$ 函数的变系数非局域非线性薛定谔方程进行了研究。研究的主要内容和上一章的内容基本类似。首先对非局域的相关方程的研究背景进行了介绍,然后通过扩展的 AKNS 系统构造了方程的 Lax 对,然后通过得到的 Lax 对得到原方程的 B\"{a}cklund变换,并在此基础上通过零解得到了原方程的单孤子解,此外还推导得出了方程的无穷守恒律。

第五章,守恒律求解程序包开发。本章节主要是开发方程守恒律的程序包,通过借助符号计算工具 Mathematica 开发一个可以自动求解前 $n$ 项守恒律的程序包。该程序包通过输入方程守恒律的迭代关系式和必要系数的通项公式以及起始项便可自动求解出方程前几组的守恒律,不仅减少了手工计算的复杂性,而且还保证计算的正确性并且提高了研究效率。












