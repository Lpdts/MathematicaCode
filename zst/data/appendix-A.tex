% !Mode:: "TeX:UTF-8"
\chapter{奇异值分解}
所有矩阵$A\in \Re^{m \times n}$可被分解成具有特殊性质的三个矩阵的成绩,即
\begin{eqnarray}
  A = USV^{T}. \label{SVD}
\end{eqnarray}
其中$U$和$V$分别是$m\times m$和$n\times n$的正交矩阵,即它们满足关系$U^{T}U = UU^{T} = I$和$V^{T}V = VV^{T} = I$。
$S$是$m \times n$对角矩阵,对角元素为$\sigma_{i},i = 1,2,\cdots,\min\{m,n\}$,其满足
\[
\sigma_{1} \geq \sigma_{2} \geq \cdots \sigma_{\min\{m,n\}} \geq 0 .
\]

称这些对角元素为$A$的奇异值(singular value),称(\ref{SVD})为矩阵$A$的奇异值分解(Singular value decomposition, SVD)。

若$A$是对称正定矩阵,它的奇异值和特征值是一致的。