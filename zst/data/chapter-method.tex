% !Mode:: "TeX:UTF-8"

\chapter{理论和方法}
\section{可积性研究}
非线性微分方程的可积性问题是孤子理论研究中的最为重要的问题之一,它不仅受到广泛的关注,而且相关的研究已有深入而系统的成果。但是要严格的给出微分方程的可积性的定义是十分困难的,就目前的研究情况而言,判断一个微分方程是否可积仍然没有一个明确而又清晰的标准。因此,在谈论微分方程可积性的时候,人们往往不会使用严格统一的定义,而是在不同条件或者情况下给出一些可以使用的可积性的概念。目前主流的可积性质主要有以下的几种,例如能够用逆散射方法求解非线性偏微分方程的 IST 可积,其它的还有 Lax 可积、Painlev\'{e} 可积、Lax 可积和 Liouville 可积等,每一种可积性质都有其对应的判断方式和计算方法。本章主要是对方程的 Painlev\'{e} 可积和 Lax 可积进行研究\upcite{syc-12}。

\subsection{Painlev\'{e} 可积性质}
逆散射方法(inverse scattering method)最早可以追溯到 20 世纪 30 年代的量子力学的研究中,到 20 世纪 50 年代由C. S. Gardner, J. M. Greene, M. Kruskal 和R. M. Miura 等人提出并用于求解 KdV 方程的积分方程法成为现今求解逆反射问题的一种标准模式。\upcite{syc-12}因为该方法在求解其它的一些非线性偏微分方程的有效性,使得其逐渐发展成为一种新的数学物理方法,并且成为了研究可积性最重要的方法之一。如果想要知道一个偏微分方程是否完全可积,可以尝试使用逆散射方法对其求解,如果能够求解,则可以认为该方程是完全可积的。然而,如何判断一个偏微分方程是否能够用逆散射方法求解仍是一个尚未解决的问题。

1981 年前后,M. J. Ablowitz,A. Ramani和H. Segur 在研究可以使用 IST 方法求解的非线性偏微分方程的相似约化时,提出了被称为 ARS 的方法:通过某种变换将偏微分方程转换为常微分方程,分析该常微分方程的奇点,就可以得到偏微分方程的结果\upcite{syc-12}。他们发现,可以用逆散射方法求解的非线性偏微分方程,经过某种形式变换后所得到的每一个常微分方程都具有 Painlev\'{e}性质。因此他们在此基础上提出一个被称为 Painlev\'{e} 的猜想,如果一个非线性偏微分方程经过某种变换可以转化成的 Painlev\'{e} 型的方程 ,那么该方程一定可以用逆散射方法求解。虽然该猜想的正确性已经得大量事实的支持,但至今都还没有清晰而系统的证明。

1983年,J. Weiss,M. Tabor和G. Carnevale 等人通过推广常微分方程的 Painlev\'{e} 性质到偏微分方程中,提出了被称为 WTC 的方法。该方法是指据变换转化成的常微分方程是否具有Painlev\'{e}性质来去判断一个偏微分方程是否具有Painlev\'{e} 的性质,并且给出了检验常微分方程是否有Painlev\'{e}性质的算法。WTC 方法是研究非线性偏微分方程可积性的一个十分有效的方法,它不仅可以判断偏微分方程是否具有Painlev\'{e}性质,而且还可以得到该方程的双线性形式、Lax 对、B\"{a}cklund 变换等,并进一步求得该方程各种不同形式的解。大量的研究结果证明了该方法的有效性,而且许多的研究者通过使用 WTC 方法对不同的偏微分方程进行研究并取得了不错的成果\upcite{syc-8}。

一个偏微分方程具有 Painlev\'{e} 可积性质,当且仅当该方程的解关于非特征流动的奇异流形是“单值”的。一般情况下,对一个非线性偏微分方程进行研究首先需要对其检验是否具有  Painlev\'{e} 性质。一般来说如果一个方程具有 Painlev\'{e} 性质,那么人们就会认为该方程是完全可积的。完全可积的方程具有很多重要的性质,例如完全可积的方程大部分都可以用反散射方法求解,并且存在 Lax 对、 B\"{a}cklund 变换、N 孤子解以及无穷守恒律等性质。下面给出 WTC 方法具体的求解步骤。

按照 WTC 方法的定义,并以  1+1 维常系数非线性偏微分方程为例,
\begin{equation}
u_t=P(u),\label{method-equ}
\end{equation}
其中 $u$ 是关于空间变量 $x$ 和 时间变量 $t$ 的函数,$P(u)$ 是 $u$ 以及 $u$ 关于 $x$ 的各阶偏导数的函数,如果判断该方程是否具有  Painlev\'{e} 性质则需要以下几个步骤:

(一) 主项分析求主项

首先假设方程(\ref{method-equ})的解具有如下的形式
\begin{equation}
u(x,t)=\phi^p(x,t) u_0(x,t),\label{method-lead}
\end{equation}
其中 $p$ 是一个负整数, $u_0(x,t)$ 恒不等于 0,并且它在流行 $\phi(x,t)=0$ 的领域内解析。将上述等式 (\ref{method-lead})代入方程 (\ref{method-equ}),由 $\phi$ 的最低次幂项可以求出 $p$ 的值和 $u_0(x,t)$ 的值。

(二)确定共振点和递推关系

将上面给出的 1+1 维非线性方程(\ref{method-equ})的解展开为如下的广义 Laurent 级数形式
\begin{equation}
u(x,t)=\phi^p(x,t)\sum_{j=0}^\infty u_j(x,t) \phi^j(x,t),\label{method-expand}
\end{equation}
其中 $u_j(x,t)(j=0,1,2,\cdots)$ 是 $\phi(x,t)=0$ 所决定的非特征流动奇异流形的领域内的解析函数。接着将以上的展开式 (\ref{method-expand}) 代入方程 (\ref{method-equ}),并令 $\phi$ 的各次幂的系数为零,求解方程组逐步推导出 $u_j$ 的递推关系式,并且可以求解得到原方程的共振点。

(三)求解相容条件

检查第二步骤中所求出的不同共振点的个数是否与方程 (\ref{method-equ}) 的最高阶数相同,如果相同则继续检查共振点是否都是正整数(除 0 和  1 外),进而求出每个共振点下的相容条件是否成立,如果成立则称该方程具有 Painlev\'{e} 性质。

目前已经有论文开源了用于分析微分方程 Painlev\'{e} 性质的 Mathematica 程序包,该程序包也是使用 WTC 方法,但是由于 Mathematica 软件目前的局限性,该程序只能用于一些简单的微分方程,对于稍微复杂的微分方程还是需要人工的使用 WTC 方法进行一步一步的分析。下面以标准的 Burgers 方程为例,来去简要的介绍 WTC 方法如何判断方程是否具有 Painlev\'{e} 性质以及具体的使用过程。

标准的 Burgers 方程具有以下的形式
\begin{equation}
u_t + u u_x - u_{xx}=0.\label{p-1}
\end{equation}
首先按照步骤一进行主项分析,将表达式 (\ref{method-lead}) 代入方程 (\ref{p-1}) 求出结果
\begin{equation}
p=-1, \quad u_0(x,t)=-2 \phi_x.
\end{equation}
然后再求出递推关系式
\begin{align}
u&=\phi^{-1}(x,t)\sum_{j=0}^{\infty}u_j(x,t)\phi^j(x,t)=\sum_{j=0}^{\infty}u_j(x,t)\phi^{j-1}(x,t),\label{p-2}\\
u_t&=\sum_{j=0}^{\infty}[u_{j,t}\phi^{j-1}+(j-1)u_j\phi_t\phi^{j-2}]=\sum_{j=0}^{\infty}[u_{j-2,t}\phi^{j-1}+(j-2)u_{j-1}\phi_t]\phi^{j-3} \\
u_x&=\sum_{j=0}^{\infty}[u_{j,x}\phi^{j-1}+(j-1)u_j\phi_x\phi^{j-2}]=\sum_{j=0}^{\infty}[u_{j-1,x}+(j-1)u_{j}\phi_x]\phi^{j-2}, \\
u_{xx}&=\sum_{j=0}^{\infty}[u_{j,xx}\phi^{j-1}+2(j-1)u_{j,x}\phi_x\phi^{j-2}+(j-1)u_j\phi_{xx}\phi^{j-2}+(j-1)(j-2)u_j\phi_x^2\phi^{j-3}]\nonumber\\
&=\sum_{j=0}^{\infty}[(j-1)(j-2)u_{j}\phi_x^2+2(j-2)u_{j-1,x}+(j-2)u_{j-1}\phi_{xx}\phi^{xx}+u_{j-2,xx}]\phi^{j-3}, \\
uu_x&=\phi^{-3}\sum_{j=0}^{\infty}u_j\phi^j\cdotp \sum_{j=0}^{\infty}[u_{j-1,x}+(j-1)u_j\phi_x]\phi^j =\phi^{-3}\sum_{j=0}^{\infty}\sum_{k=0}^{j}[(u_{k-1,x}+(k-1)u_k\phi_x)u_{j-k}]\phi^j\nonumber\\
&=\sum_{j=0}^{\infty}\sum_{k=0}^{j}[(u_{k-1,x}+(k-1)u_k\phi_x)u_{j-k}]\phi^{j-3}.\label{p-3}
\end{align}
得到方程 (\ref{p-1}) 的主项分析结果后,再将 (\ref{p-2})-(\ref{p-3}) 式代入原方程,整理可得如下关系式
\begin{equation}
(j+1)(j-2)u_j=F_j,\label{p-4}
\end{equation}
其中 $F_j$ 的表达式为
\begin{align}
F_j=&\frac{1}{\phi_x^2}\big[ u_{j-2,t} + (j-2) u_{j-1}\phi_t +\sum^{j-1}_{k=1} u_{j-k} [ u_{k-1,x} +
(k-1) u_k \phi_x] -u_{j-2,xx} \nonumber\\
&- (j-2)(u_{j-1,x} \phi_x +u_{j-1} \phi_{xx}) \big],  \qquad (j\geq 0, \mbox{ 因为当定义 }  j<0 \mbox{ 时 } u_j=0 )
\end{align}
从上述关系式式 (\ref{p-4}) 可看出,当 $j = −1$ 和 $j = 2$ 时,$u_j$ 无法确定,因此方程 (\ref{p-1}) 的共振点为 −1 和 2。 $j = −1$ 对应函数 $\phi(x,t)$ 的任意性,因此也就是当 $j = 2$ 时只会出现一个相容条件。将 $j = 1$ 和 $j = 2$ 代入方程 (\ref{p-4}),整理可得
\begin{align}
& j=1: \quad \phi_t +u_t \phi_x=\phi_{xx},\label{p-5}\\
& j=2: \quad \frac{\partial}{\partial x}\big( \phi_t+u_t \phi_x - \phi_{xx} \big)=0. \label{p-6}
\end{align}
可以看出,当 $j = 2$ 是也就是(\ref{p-6}) 式 是方程 (\ref{p-5}) 对 $x$ 取微分的结果,即相容条件满足,Burgers 方程 (\ref{p-1}) 式 具有 Painlev\'{e} 性质。上述过程就是使用 WTC 方法对一个微分方程检测是否具有 Painlev\'{e} 性质。

应用 WTC 方法,人们证明了许多非线性偏微分方程具有 Painlev\'{e} 性质,或称是  Painlev\'{e} 可积的。但是可以看出当 $F_j$ 十分复杂是,检验 Painlev\'{e} 性质就会存在一定的困难。为此,到 1984 年时,Kruskal 等人对 WTC 方法进行了一定的简化,即提出了 $\phi$ 函数特殊的简化形式
\begin{equation}
  \phi(x,t) = x + \psi(t) = 0.
\end{equation}
此形式大大简化了计算的复杂性,因此也被称作为 Kruskal 简化形式。

\subsection{Lax 可积}
Lax 对 (Lax pair) 本质上是逆散射方法的一种推广。当 Gardner 等人使用逆散射方法研究 KdV 类的方程时,有人认为这可能只是某些 KdV 类方程之间的一种巧合。但是后来的研究表明,逆散射方法可以用来求解许多其它类的非线性偏微分方程,其中一个比较重要的进展是在 1968 年时 Lax 将这种方法理论化,给出一个比较普遍的处理模式。而到了 1972 年,Zakharov 和 Shabat 将 Lax 的思想用于求解非线性薛定谔类的方程获得了成功,从而进一步的证明了逆散射方法确实不是一种幸运的巧合\upcite{syc-12}。

如果一个非线性偏微分方程可以由多个线性可积表达式表示,那么就称该方程是 Lax 可积的。Lax 对不仅推广了逆散射方法的使用范围,而且还能在此基础上进一步推导对应方程的 B\"{a}cklund 变换、 Darboux 变换和无穷守恒律等重要性质。

根据 Lax 对的方法,求解一个非线性偏微分方程
\begin{equation}
  \partial^m_t u = F(x, t, u, u_x, u_t, \cdots),
\end{equation}
可简要分为以下三个步骤:

1. 找到原方程的一个合适的本征值问题
\begin{equation}
  L\psi = -\lambda \psi,
\end{equation}
其中 $L$ 是一个线性算子,并且与 $u$ 相关。

2. 证明本征值 $\lambda$ 与 $t$ 没有关系,即满足
\begin{equation}
  \frac{d\lambda}{dt} = 0.
\end{equation}

3. 再找到一个合适的线性算子 $M$,$M$ 也与 $u$ 有关,并且使得
\begin{equation}
  \psi_t = M\psi
\end{equation}

以方程方程 (\ref{method-equ}) 为例, 对于该方程 的解 $\phi$, 如果存在矩阵 $U, V$ 满足
\begin{equation}
  \Phi_x=U \Phi, \qquad \Phi_t=V \Phi, \label{method-Lax}
\end{equation}

其中 $\Phi=(\phi_1, \phi_2, \cdots)^T, \Phi_x=(\phi_{1x}, \phi_{2x}, \cdots)^T, \Phi_t=(\phi_{1t}, \phi_{2t}, \cdots)^T$,并且 $\phi_1, \phi_2$ 等是关于 $x$ 和 $t$ 的函数, $T$ 表示向量的转置。 将条件 $\Phi_{xt}=\Phi_{tx}$ 作用到 (\ref{method-Lax}) 式可以得到如下方程
\begin{align}
U_t-V_x+UV-VU=0.
\end{align}
该等式也称相容条件。如果存在矩阵 $U, V$ 使上述等式成立并且与方程 (\ref{method-equ})等价,那么该方程就是  Lax 可积的。$U, V$ 则被称为方程的 Lax 对。Lax 对的构造方法主要有  AKNS 方法和双线性法,但是迄今为止还没有一个比较好的方法来去判断一个非线性偏微分方程是否 Lax 可积,如何构造或者得到一个方程的 Lax 对任然是一个比较困难的问题,大多数的情况只能对已有方程的 Lax 对进行改造或者是直接凭借经验和直觉。因此对偏微分方程的 Lax 可积性的研究仍是一项广大学者关注的重点内容。

\subsection{守恒律}
提到守恒律人们一般会想到物理学中的质量守恒、能量守恒和动量守恒等,而本小节中的守恒律是数学中的概念。当现实世界中的一种现象可以被一个偏微分方程表示,并且该偏微分方程的解及其对空间变量或时间变量的各阶导数均满足某稳定的关系,则该偏微分方程具有守恒律。具体来说,对于一个非线性偏微分方程
\begin{align}
u_t = H(u) =  H(u, u_x, u^2,u_{xx}, \cdots)\label{inf02}
\end{align}
若存在该方程的解  $u(x, t)$ 及其对空间变量的各阶导数 $\dfrac{\partial^pu}{\partial x^p}(p=1,2,\cdots)$ 的函数 $T(u), X(u)$, 使得
\begin{align}
\frac{\partial T(u)}{\partial t}+\frac{\partial X(u)}{\partial x}=0. \label{inf01}
\end{align}
成立,那么我们就可以将 (\ref{inf01}) 式称为方程 (\ref{inf02}) 的一个守恒律,其中 $T(u)$ 表示守恒密度, $X(u)$ 表示 $T$ 的流量。特别地,如果 $T$ 仅只是 $u$ 及其对 $x$ 的各阶导数的多项式,不直接包含 $x, t$, 则称 $T$ 为多项式守恒密度,那么此时 (\ref{inf02}) 式是方程 (\ref{inf02}) 的一个多项式守恒律。

在物理学科中守恒律一直是一个很经典的问题。而在数学学科中,守恒律对于探索非线性方程的解具有关键性的作用。在非线性孤子领域中,随着人们研究的深入,越来越多的结果证明孤子的存在与方程的守恒律之间有着十分紧密的联系。很多非线性偏微分方程具有孤子解的同时具有无穷守恒律的性质。虽然目前尚无完全的证据证明两者之间的必然联系,但无论如何,非线性偏微分方程的守恒律研究都具有十分重要的意义。


\section{B\"{a}cklund 变换}
除了之前介绍逆散射方法以外,还有其它的方法可以用来求解非线性偏微分方程,B\"{a}cklund 变换法就是其中的一种。B\"{a}cklund 变换是指两个偏微分方程的解之间存在的关系表达式,也就是说,知道其中一个偏微分方程的解,可以通过 B\"{a}cklund 变换得到另外一个偏微分方程的解\upcite{syc-12,so-1,so-2}。如果这两个偏微分方程刚好是同一个方程,那么这时 B\"{a}cklund 变换也被称为自 B\"{a}cklund 变换。对于自 B\"{a}cklund 变换而言,如果已知一个解,那么理论上通过自B\"{a}cklund 变换关系式可以迭代除无穷个解,只是随着迭代次数的增加,解的形式会越来越复杂,不过这种方法仍然是一个可行有效的方法。因此,B\"{a}cklund 变换法是求解非线性偏微分方程的一个很重要的方法,并已经被广泛的使用。

B\"{a}cklund 变换最早由  A. V. B\"{a}cklund 提出。早在 1875 年,瑞典的一位数学家  B\"{a}cklund 在研究负常数曲率曲面(喇叭形曲面)时,发现了非线性的  Sine-Gordon 方程 $u_{xt}=\sin u $ 的两个解 $u$ 和 $u^*$ 之间有以下的关系:
\begin{eqnarray}
(\dfrac{u+u^*}{2})_{x} & = a\ \sin \dfrac{u-u^*}{2}  , \label{method-BT-151}
\\
 (\frac{u-u^*}{2})_{t} &= a^{-1}\ \sin \frac{u+u^*}{2} , \label{method-BT-152}
\end{eqnarray}
其中 $a\neq 0$ 且为常数。方程(\ref{method-BT-151}) 和 (\ref{method-BT-152}) 就是 Sine-Gordon 方程的自 B\"{a}cklund 变换表达式。通过该变换可以很容易的求出 Sine-Gordon 方程的解。例如取平凡解 $u=0$,则由方程组 (\ref{method-BT-151}) 和 (\ref{method-BT-152}) 可以求出 Sine-Gordon 方程的另外一个解:
\begin{eqnarray}
u^*=4\arctan (e^{a\,x+a^{-1}\,t+\alpha} ) ,
\label{method-BT-153}
\end{eqnarray}
其中 $\alpha$ 是积分常数。而(\ref{method-BT-153}) 式也被称为 Sine-Gordon 方程的扭结孤子解。 理论上,如果将此解 \eqref{method-BT-153} 代入自 B\"{a}cklund 变换表达式 (\ref{method-BT-151}) 和 (\ref{method-BT-152}),则又可以得到 Sine-Gordon 方程的新解,重复上述过程,则可以求出更多的解。然而在实际计算中,迭代次数增加会导致计算会变得越来越复杂,以至于无法获得新的解,这限制了 B\"{a}cklund 变换的使用。随着研究的深入,互换定理和非线性叠加公式的发现使得该变换的作用得以充分的发挥。

以上主要介绍了在求解非线性偏微分方程中 B\"{a}cklund 变换的重要性,而 B\"{a}cklund 变换的求法有很多,常见的有:(1) 可积性条件;(2) Painlev\'{e} 截断方法;(3) AKNS$\footnote{Ablowitz, Kaup, Newell, Segur}$ 系统及逆散射方法;(4) 双线性形式;(5) 齐次平衡法。本文涉及的方法主要有AKNS系统及逆散射方法。


%当 $X(u)$ 是空间变量 $x$ 的周期函数或在无穷远处 ($x\textrightarrow \pm \infty$) 迅速趋于零时,由 (\ref{inf01}) 关于 $x$ 积分得
%\begin{align}
%\dfrac{d}{dt}\int_{-\infty}^{\infty}T(u)dx=0\nonumber
%\end{align}
%或
%\begin{align}
%\int_{-\infty}^{\infty}T(u)dx=I=const\label{inf03}
%\end{align}
%与时间 $t$ 无关,也就是在变化过程中始终保持守恒,也称方程 (\ref{inf03}) 为守恒积分或运动积分或运动常数。若 $T(u)$ 对一切 $u$ 可表为 $T(u)=F_x(u)$,则称守恒密度是平凡的,因为这时自然有
%\begin{align}
%(F_{x})_t+(-F_t)_x=0.
%\end{align}

%守恒律(如能量守恒、质量守恒定律)历来是物理学中研究的中心课题之一,在数学中借助运动常数,对偏微分方程的解做出先验估计,这些估计是方程解的存在性、唯一性及稳定性的核心和关键,在计算数学中较好的稳定差分格式,也用到守恒律,更有意思的是,越来越多的事实表明:孤立子的存在与无穷多个守恒律的存在是有密切联系的,具有孤子解的非线性发展方程,无论是 KdV 方程还是薛定谔方程,一般都具有无穷守恒律这一性质,只是很多方程的守恒律至今还没有找到,因此研究守恒律以及是否存在无穷多守恒律是具有现实意义与理论意义的\upcite{keji}。

