% !Mode:: "TeX:UTF-8"

% 学院中英文名,中文不需要“学院”二字
% 院系英文名可参见
% http://ev.buaa.edu.cn/education/index.php?page=department
\school
{计算机}{School of Computer Science and Engineering}

% 专业中英文名
\major
{计算机科学与技术}{Computer Science and Engineering}

% 论文中英文标题
\thesistitle
{基于符号计算的非线性模型方程的解析研究}
%{这里是长长的长长的长长的长长的长长的长长的长长的长长的长长的副标题}
{ }
{Analytical research on nonlinear model equation with symbolic computation}

% 作者中英文名
\thesisauthor
{张舒涛}{Zhang Shutao}

% 导师中英文名
\teacher
{张玉平}{Zhang Yuping}
% 副导师中英文名
% 注:慎用‘副导师’,见北航研究生毕业论文规范
%\subteacher{盛浩}{Sheng Hao}

% 中图分类号,可在 http://www.ztflh.com/ 查询
\category{O141}

% 本科生为毕设开始时间;研究生为学习开始时间
\thesisbegin{\ \ \ \ \ \ \ \ }{\ \ \ \ }{\ \ \ \ }

% 本科生为毕设结束时间;研究生为学习结束时间
\thesisend{\ \ \ \ \ \ \ \ }{\ \ \ \ }{\ \ \ \ }

% 毕设答辩时间
\defense{\ \ \ \ \ \ \ \ }{\ \ \ \ }{\ \ \ \ }

% 中文摘要关键字
%\ckeyword{三维测量,亮度感知,高动态场景恢复,图像融合}

% 英文摘要关键字
%\ekeyword{Three-Dimensional Measurement, Highlight Perception, High Dynamic Range Recovery, Image Fusion}
