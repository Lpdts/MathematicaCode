\definecolor{dkgreen}{rgb}{0,0.6,0}
\definecolor{gray}{rgb}{0.5,0.5,0.5}
\definecolor{mauve}{rgb}{0.58,0,0.82}

\lstset{frame=tb,
  language=Mathematica,
  aboveskip=3mm,
  belowskip=3mm,
  showstringspaces=false,
  columns=flexible,
  basicstyle={\small\ttfamily},
  numbers=left,
  numberstyle=\tiny\color{gray},
  keywordstyle=\color{blue},
  commentstyle=\color{dkgreen},
  stringstyle=\color{mauve},
  breaklines=true,
  breakatwhitespace=true
  tabsize=3
}

\chapter*{附录 A\markboth{附录 A}{}}
\addcontentsline{toc}{chapter}{附录 A}
\begin{lstlisting}[language=Mathematica]
(*
	c1: 通项公式 cn 的首项内容
	d1: 通项公式 dn 的首项内容
	coe1: 对应系数的函数列表
	coe2: 对应系数的函数其它形式(例如共轭)列表
	n: 表示第 n 组的结果
*)
getGeneralTerm[c1_, d1_, coe1_, coe2_, n_]:=Module[{temp1, temp2, clist, dlist},
	If[n <= 0, Print["n输入不合法"],
	If[Length[coe1]!= Length[coe2], Print["系数列表长度不符合要求"],
	(*通项公式 C 的列表*)
	clist = List[];
	(*通项公式 d 的列表*)
	dlist = List[];
	(*列表 clist, dlist 的填充项,为保持长度统一*)
	clist = Append[clist, 0];
	dlist = Append[dlist, 0];
	(*列表 clist, dlist 的首项内容*)
	clist = Append[clist, c1];
	dlist = Append[dlist, d1];
	For[i=2, i <= n+3, i++,
  temp1 = 1/(2aI)(D[clist[i],x]+a$\sum_{m=0}^{i-1}$(clist[[m+1]]clist[i-m]coe1[[1]]coe2[[2]]
            +clist[[m+1]]dlist[[i-m]]coe2[[1]]coe1[[2]]));
  temp2 = 1/(2aI)(D[dlist[i],x]+a$\sum_{m=0}^{i-1}$(dlist[[m+1]]dlist[[i-m]]coe2[[1]]coe1[[2]]
            +dlist[[m+1]]clist[[i-m]]coe1[[1]]coe2[[2]]));
	clist = Append[clist, temp1];
	dlist = Append[dlist, temp2];]
	Print["第 ".n." 项 c 是:"]
	Print[Simplify[clist[[n+1]]]];
	Print["第 ".n." 项 d 是:"]
	Print[Simplify[dlist[[n+1]]]];
]]]
\end{lstlisting} 