% !Mode:: "TeX:UTF-8"
\chapter{绪论}
\section{研究背景}
\subsection{符号计算}


\subsection{非线性科学与怪波}


\section{研究现状}
近年来以非线性偏微分方程为主要研究对象的孤立子理论发展十分迅速,形成了许多有效的研究方法,例如反散射法 (IST)、B\"{a}cklund 变换法、Painlev\'{e} 有限展开法、Lie 群法、Hirota 双线性法和 Darboux 变换法等。伴随着各式各样解析方法的出现,许多新的方程和新的解都不断的被发现和利用。下面就关于孤立子的解析方法的发展做一些简单的介绍。

1967 年, Gardrier 等人运用




\section{主要研究内容}

本论文的研究内容主要是利用符号计算工具 Mathematica 对两类方程进行了研究,其中第一类是变系数的 Sasa-Satsuma 方程,而第二类是空间局域的变系数 Sasa-Satsuma 方程。对方程的解析研究内容主要包括  Painlev\'{e} 可积性质分析、Lax 对、B\"{a}cklund 变换、孤子解、无穷守恒律以及孤子解传播的模拟等。进而在对方程守恒律等性质研究方法充分了解的基础上,编写基于 Mathematica 程序包对守恒律的求解进行自动化,减轻了人工计算的难度,同时在一定程度上践行了数学机械化的思想。

本文主要进行以下几个方面的研究:

第一部分为变系数的 Sasa-Satsuma 方程的解析研究及孤立波传播模拟。本部分所研究的  Sasa-Satsuma 方程同时包含 6 项关于 $t$ 函数的变系数,这使得在求解方面会变得有些复杂,但是求得解范围也更广,也能描述更多的非线性现象。首先,本部分研究方程的 Painlev\'{e} 可积性质,并在其中一组相容条件下扩展了 AKNS系统求得了方程的 Lax 对。在所求得的 Lax 对的基础上,得到了方程的自 B\"{a}cklund变换,进而根据该变换得到了方程的单孤子解。此外还引入Riccati 辅助变量推导了方程的无穷守恒律,列出并且验证了前几组的守恒律。最后借助之前所求得的单孤子解,进行了孤立波的模拟并分析了方程各项的系数对于孤立波传播的影响。

第二部分为局域变系数的 Sasa-Satsuma 方程的解析研究。本部分是在之前研究的基础进行更加深入的探索,使用的方法以及研究的性质和之前的基本相同,但是由于增加了空间局域的条件,求解方面就变得更加的复杂以及更有难度。本部分所研究的方程包含 7 项 关于 $t$ 函数的变系数,并且空间变量增加了局域性的条件,这使得方程求解与之前的研究相比变得更加复杂。本部分首先构造并求得了方程的 Lax 对,再通过 Lax 对得到了方程的自 B\"{a}cklund变换,然后结合种子零解得到了方程的单孤子解,并且研究了方程的无穷守恒律。最后借助所得到的孤子解,利用 MAthematica 进行了模拟与分析。

最后一部分为自动求解方程守恒律程序包的开发。本部分是在前面两部分研究的基础上,借助符号计算工具 Mathematica 开发一个可以自动获得方程守恒律的程序包。该程序包通过输入方程守恒律的迭代关系式和必要系数的通项公式以及起始项,自动求解出方程前几组的守恒律,不仅减轻了手工计算的难度,而且还能保证结果的正确性,提高了研究计算效率。

\section{研究意义}
随着计算机科学及其相关技术的快速发展,符号计算系统越来越强大,如何将符号计算工具应用于更多的科学工程领域并帮助其发展变得更有意义。作为数学、物理、计算机等众多交叉学科都涉及到的非线性相关问题一直是学术界的研究的重点和难点,尤其是对用来描述现实世界中诸多非线性现象的非线性偏微分方程研究更是具有重要的意义。

非线性偏微分方程的研究最大的困难之一在于解析解的获得十分困难,不像线性的常微分方程具有相较统一的求解方式,不同的非线性偏微分方程一般有着不同的求解方法,而这些方法又往往比较复杂。近年来,大量的经典书籍和科研文献都致力于将已有方法推广到更多的非线性偏微分方程或者是获得更多新的用于求解的方法。因此,解析研究将直接影响非线性偏微分方程的理论研究和发展。

如何更好的利用符号计算系统工具 Mathematica 来对相关的非线性偏微分方程的可积性以及解析解等性质进行研究是本文研究的重点。随着对非线性偏微分方程深入研究,所研究的方程以及其计算过程变得越来越复杂,人工计算变得越来越不可靠,因而必须借助相关的符号计算系统。利用 Mathematica 对非线性偏微分方程的解析性质进行研究,对孤立波进行模拟,并将方程的某些性质求解进行符号化,编写出可靠实用的 Mathematica 程序包是本文的主要研究内容。此外本文的研究成果对某些非线性偏微分方程的研究会起到一定的促进作用。

\section{论文结构安排}
本论文总共包含五个章节,其中第一、二章节主要介绍论文的研究背景和所需要的基本的数学理论和方法;第三、四、五章节是论文的主要研究内容。论文详细的安排如下:

第一章,绪论。本章首先对作为论文研究重点的符号计算和非线性偏微分方程的背景知识进行了简单的介绍,进而对孤子研究领域和符号演算的研究发展进行了总结和概括,然后阐述了本论文的主要研究内容和研究意义,最后给出了论文内容的结构安排。

第二章,相关的数学理论和方法。本章对之后几个章节的研究中所用到的基本的数学方法和理论进行了概述,其中所使用的数学方法包括 Painlev\'{e} 可积性质的分析、Lax可积、B\"{a}cklund变换、无穷守恒律和孤子解的求解。这些内容都是后续对非线性偏微分方程的研究中所必须使用到的。

第三章,变系数的 Sasa-Satsuma 方程。本章节对一个含有 6 个关于 $t$ 函数的变系数非线性薛定谔方程进行了研究。研究内容主要包括方程的 Painlev\'{e} 可积性质、Lax 对、单孤子解、无穷守恒律和孤立波的模拟。详细的内容如下:首先对于变系数的 Sasa-Satsuma 方程相关的研究背景进行了阐述,进而通过  Painlev\'{e} 检测得到了方程的  Painlev\'{e} 可积条件,在此基础上通过扩展的 AKNS 系统构造了方程的 $3 \times 3$ 阶的 Lax 对,然后基于得到的 Lax 对得到了原方程的 Riccati 形式的自 B\"{a}cklund变换,并使用种子解和该变换得到了原方程的单孤子解,另外又通过 Lax 对推导了方程无穷守恒律的关系式,并列举给出了前三组的守恒律,最后借助得到的孤子解,对孤立波进行模拟并分析了原方程系数对孤立波的波形和传播过程产生的影响。

第四章,变系数的非局域的 Sasa-Satsuma 方程。本章节是在前一章的基础上进行研究的,也可以说是前一章研究内容的延续。本章节对一个含有 7 个关于 $t$ 函数的变系数非局域非线性薛定谔方程进行了研究。研究的主要内容和上一章的内容基本类似。首先对非局域的相关方程的研究背景进行了介绍,然后通过扩展的 AKNS 系统构造了方程的 Lax 对,然后通过得到的 Lax 对得到原方程的 B\"{a}cklund变换,并在此基础上通过零解得到了原方程的单孤子解,此外还推导得出了方程的无穷守恒律,最后使用 Mathematica 对方程的孤子解进行模拟,分析了方程的变系数对孤立波的传播所产生的影响。

第五章,守恒律求解程序包开发。本章节主要是开发方程守恒律的程序包,通过借助符号计算工具 Mathematica 开发一个可以自动求解前 $n$ 项守恒律的程序包。该程序包通过输入方程守恒律的迭代关系式和必要系数的通项公式以及起始项便可自动求解出方程前几组的守恒律,不仅减少了手工计算的复杂性,而且还保证计算的正确性并且提高了研究效率。












