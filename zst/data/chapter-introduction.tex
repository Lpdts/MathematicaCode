% !Mode:: "TeX:UTF-8"
\chapter{绪论}
\section{研究背景}
\subsection{符号计算}
先进工具的发现会推动人类文明的重大变革,现代电子计算机的出现与飞速发展逐渐改变了我们的工作和生活方式,对于与计算机科学和技术紧密相关的数学,它的发展和进步必然面临着史无前例的机遇与挑战。如何更好的将计算机与数学融合以进行高效的演算、推理、研究、教学和应用?如何让数学为计算机科学和技术的发展起到更大的推动作用?符号计算应需而生。

计算机,顾名思义是用于计算的机器。正因为有了这样的工具,数学的计算过程才变得更加高效。数学计算分为符号计算和数值计算两种,其中数值计算是指浮点数的运算,在现有常用的计算机编程语言如 C 和 Java 中很容易实现确定精度的浮点运算。以双精度浮点运算为例:
\begin{align}
0.1+0.2==0.30000000000000004.
\end{align}
显然,这样的计算是带有误差的。算法的稳定性在使用数值计算时是必须要考虑的,输入数据的微小扰动可能会引起输出的大幅波动。因此,在某些对精确度结果要求很高的工程和科学问题,如公钥密码学或一些非线性问题中, 当使用纯粹的数值方法无法达到预期时便会引入符号计算。数值计算处理的是浮点数,会有误差,符号计算处理的是具有特定含义的符号,可以是数字、表达式,甚至是图形与程序,主要研究的是精确计算\upcite{fuhao}。以下面的计算为例:
\begin{align}
\frac{3x^2-27}{x+3}=3x-9.
\end{align}
上式是精确的符号计算,没有误差。符号之间的运算可以十分复杂,如何在电子计算机上表示和处理符号对象,设计和实施用于符号计算的有效算法和软件系统便是符号计算的主题,它们无疑会成为解决科学与工程中的各种数学计算问题的强有力的工具。

符号计算在计算数学中也被称为计算机代数或代数计算,是一个对处理数学符号、数学表达式和其它数学对象的算法和软件进行研究和开发的科学领域。虽然符号计算应该是科学计算的一个子领域,但它们通常被认为是不同的,因为科学计算一般是基于数值计算,而符号计算强调的是对包含变量的表达式的精确计算。因而对于要求高精度的数学或者其它领域来说,符号计算可以为它们的理论研究大大提高效率,与传统的数值计算研究方式有很大区别\upcite{sym-com14}。 目前符号计算正在蓬勃发展,是当下数学科学研究的焦点和前沿。

执行符号计算的应用程序软件被称为计算机代数系统,系统这个术语表示计算机代数应用程序的复杂性,至少包括计算机中表示数学数据的方法、用户编程语言(通常不同于实现该应用程序的编程语言)、专用的存储器管理器、用于数学表达式的输入输出的用户界面和执行常规操作的大量例程,如表达式的简化,使用链式规则的分化,多项式因式分解和不定积分等等。现在最广为人知的计算机代数系统是 Wolfram Research 的 Mathematica 和 Maplesoft 的 Maple。Mathematica 结合数值和符号计算引擎、编程语言、图形系统和文本系统于一身,其功能处于世界先进水平\upcite{sym-com16,sym-com17,sym-com18}。自 1988 年 Mathematica 发布至今,大大促进了计算机在诸多领域的应用与发展,有人说 Mathematica 的出现是现代科技计算的开端。Mathematica 的基本理念是设计一个统一并且连贯的方法来开发适用于各种需求的科技计算的系统,为达成这一目标,Mathematica 发明了一种新的基于符号的计算机语言——Wolfram。
%当Mathematica1.0版发布时,《纽约时代报》评论道:“这个软件的重要性不可忽视”;紧接着《商业周刊》又将Mathematica 评比为当年十大最重要产品。在科技界,Mathematica被形容为智能和实践的革命。
Wolfram 语言是 Mathematica 中的编程语言,支持多种程序设计模式,包括函数式和过程式等,并提供用户以 Package 形式开发程序包。到目前为止,Mathematica 已经发布了版本 11,支持多种语言,是一款十分强大的符号演算工具。

\subsection{非线性科学与怪波}
非线性现象在生活中十分常见,专门研究非线性现象的非线性科学在许多自然科学和社会科学中都有广泛的应用,是20世纪自然科学的重要发现之一\upcite{back-1}。随着自然科学和计算机技术的发展,数理化等基础学科不仅仅是纵向发展,它们与新技术的结合, 加之利用计算机作为研究工具,推出了许多涉及多个领域的交叉学科,非线性科学和符号计算都在其中,由此可见,学科划分和研究方法都发生了很大变化\upcite{nws97}。

自1950年开始,人们对非线性现象的研究取得了一系列丰硕的成果,尤其是在1965年,美国数学家 N. J. Zabusky 和 M. D. Kruskal 利用电子计算机进行数值分析时发现了孤立子的存在\upcite{NJZ65},此后不久,在1967年, C. S. Gardner、J. M. Greene、M. D. Krusal 和 K. M. Miurat 提出了逆散射方法用于求解 KdV 方程。正是这些优秀学者的研究成果,给非线性偏微分方程的研究带来了新思路,促使该研究领域得到了长足的发展。变系数非线性薛定谔方程是非线性领域最重要的方程之一, 它可以应用于非线性光学纤维\upcite{SRK07, BK09}、海洋学\upcite{MGO05}、经济学\upcite{Y11}、超流体\upcite{EGK09}等多个领域\upcite{RL95},多个变系数项可以更加准确地描述一些更复杂的非线性现象,所以变系数非线性方程被许多学者广泛研究。

怪波最初发现于海洋表面,是一种突然出现的十分陡峭的海洋波,与孤立子类似,是一种特殊波,也被称为异常波和巨波,同时物理学家将光学中出现的一种亮度很大的斑也称为怪波。怪波的振幅很大,波谷很低,波峰很高,一般来说,怪波的波峰会在波谷前后出现,不过也有可能出现一系列连续的波峰。
不同于台风和地震引发的海啸可以提前被预测,怪波通常是突然出现,没有规律,无法预测,海面突然出现很深的沟或者很高的波,造成了很多航海灾难,造成巨大的破坏\upcite{SG76}。早从1960年起,海洋学家开始研究怪波,随着怪波逐渐在各个学科领域被发现,吸引了一大批学者研究怪波描述的现象,促成了一个活跃的研究方向。

\section{研究现状}
1967年,Gardrier 等人发明了一种新的求解 KdV 方程的方法,主要是利用量子力学中的正散射问题和反散射问题之间的关系,也就是 Schr\"{o}dinger 方程的特征值问题及其反问题之间的关系,即找一种变换,将方程转换为线性可积分方程,求解该方程,得到的解就是 KdV 方程初值问题的解,这种方法被称为逆散射方法或非线性傅里叶方法\upcite{csg67}。逆散射方法的思路和概念与传统求解方法有很大区别,为数学和其它学科的研究方法提供了新的思路,产生了深远的影响。因该方法的有效性和重要性,Lax 将逆散射方法进行系统的整理并加以拓展,使它适用于不同类型的非线性偏微分发展方程的初值问题的求解,后被广泛应用。

近年来,对非线性发展方程的怪波研究得到了快速发展,有效的研究方法也逐渐形成体系。怪波在海洋学、超流体、光学纤维和经济学等诸多领域内都存在,是非线性研究领域的重要课题之一。随着怪波现象逐渐出现在公众视线中,广大学者对怪波的研究越来越多,取得了许多不错的研究成果。目前求解非线性偏微分方程的怪波解有反散射方法\upcite{BT13}、代数几何法\upcite{DM13}、Darboux 变换法\upcite{HZW13}和 Hirota 双线性方法\upcite{OY12,OY13}等具体方法,这些解法一方面解决了很多以前解决不了的问题,另一方面发现了很多有意义的新解,大大促进了怪波研究的发展。

许多学者对怪波进行了大量的研究:Akhmediev 教授小组对 $(1+1)$ 维的非线性薛定谔方程 $(NLS)$ 的怪波进行了较全面的分析\upcite{AAT09,AAS09},指出怪波是 “Ma 解” $(MS)$或“ Akhmediev 呼吸子”$(Abs)$ 的极限情形,实际上是一种非奇异的有理解;Xu 、He、Wang 以及 Porsezian 利用 Darboux 变换得到许多 $(1+1)$ 维高阶薛定谔型方程的怪波解\upcite{XH12, WH13}。但现有的文献对高维薛定谔方程的怪波解研究甚少。直到最近,Y. Ohta 和杨建科教授利用 Hirota 双线性方法得到 $(2+1)$ 维 DS-I 和 DS-II 方程的 Grammian 解,再利用算子理论将其转化为非奇异的有理解,从而将具有理分式的怪波解的方程扩展到高维的薛定谔型方程的\upcite{OY12, OY13},这使得对高维的薛定谔型方程怪波解的寻求成为非常有意义的事;M. S. Ruderman 教授预言在天空中的阿尔芬波也会有这种怪波,而阿尔芬波可以用导数非线性薛定谔 (DNLS) 方程及其变系数导数非线性薛定谔 (VCDNLS) 方程的解来描述。光学中存在自感应透明效应,而这种效应可以很好的通过非线性薛定谔 (NLS) 方程与麦克斯韦-布洛赫 (MB) 方程的耦合系统来描述。

怪波的无法预测性会造成巨大损失,所以研究怪波具有十分重要的现实意义,但是由于各方程间的结构形式差异很大,对怪波的认识尚不够深入,导致对非线性偏微分方程的怪波研究十分困难,当今怪波理论的研究对象主要分为以下四类:

1. 变系数非线性偏微分发展方程。变系数非线性偏微分发展方程因其有多个变系数项,当变系数取不同值时方程结构也大不一样,所以能模拟更多较为复杂的非线性现象。随着科技的发展,符号计算代替人工计算,复杂的变系数方程的研究也实现了进一步的推进。近年来,怪波研究的对象主要集中在薛定谔方程,对于 KdV 方程的怪波研究成果比较少,因此,对变系数偏微分发展方程的怪波研究仍是重难点。

2. 高维非线性偏微分发展方程。 在现实生活和实际的工程应用中发现了很多怪波现象,产生了很多亟待研究的非线性发展方程,考虑到实际现象中多维的因素,所要研究的方程也逐步从低维变成高维。

3. 耦合非线性偏微分发展方程。耦合方程由一对相互联系的方程组成,在实际应用中,耦合的非线性发展方程应用更加广泛,尤其是在海洋学的研究中,因此,对耦合的非线性发展方程的研究更具现实意义。

4. 高阶非线性偏微分发展方程。模拟较为复杂的现象,高阶非线性偏微分发展方程比低阶方程更加精确。已有研究表明,高阶 KdV 方程比低阶 KdV 方程 模拟短波大振幅波的传播过程更加准确真实\upcite{back-8.3}。

目前,怪波理论研究广泛应用于实践工程和实际问题,随着人们对怪波的发现和研究,怪波不只出现于海洋中,光学和金融领域等出现的很多现象也可以用怪波来解释。因此,非线性科学作为一个十分重要而广泛的交叉学科,怪波的研究也具有重要的现实意义。但是
随着所研究的非线性发展方程的复杂度越来越高,解析研究越来越依托于符号计算学科的发展。
在这种情况下,许多基于符号计算软件的自动求解方程的程序包被开发出来用于简化计算。
在可积性方面,关于偏微分方程检测算法的 Maple 程序包“SPIC”和常微分方程检测算法的 Macsyma 程序包“ODEPAINLEV\'{e}” 可以
用于检测方程的 Painlev\'{e}可积性\upcite{sym-com49,sym-com50}。
2006年,Hereman 和 Baldwin 在此基础上,进一步开发了可适用于多项式形式的微分方程 Mathematica 程序包“Painlev\'{e}Test.m”,该软件包的优势是自变量和因变量的数目都不受限制,如今,该程序包广泛运用于检测方程的
Painlev\'{e} 可积性\upcite{sym-com51}。
但是目前关于其它可积性的判定如 B\"{a}cklund 变换,Lax 对等,还没有相应的程序包。

在方程求解方面,1982年,Sehwarz 在 Reduce 上开发了求解微分方程古典对称群的程序包,
1982年,Reid 等人开发了对微分方程组进行化简的 Maple 程序包,同年,Mausfield 编写了对微分方程的对称群进行自动化计算的 Maple 程序包;
1994年,李志斌开发的程序包高效求出了许多经典非线性发展方程的精确孤立子解;
进一步地,李志斌与柳银萍等人在非线性发展方程的自动求解上开展了各种工作,开发出了一系列的程序包
\upcite{sym-com52,sym-com53}。
2008年,针对扩展的 Tanh 方法的 Maple 程序包 TWS 由 Liang 等人开发出\upcite{sym-com54}。
但以上程序包的开发大多集中于方程特殊解的特定解法,开发适用性更广的求解程序包有助于非线性模型的进一步发展与研究。
\section{主要研究内容}
本文借助符号演算工具 Mathematica 对变系数 Sasa-Satsuma 方程和常系数 6 阶 KdV 方程进行研究。对方程解析研究的内容主要包括 Painlev\'{e} 分析、B\"{a}cklund 变换、贝尔多项式、$N$ 孤子解、守恒率、怪波解以及怪波传播图的模拟等,在对方程解析研究方法充分了解的基础上,编写 Mathematica 程序包对求解怪波解进行自动化,减少了手工计算的复杂度,同时在一定程度上践行了数学机械化思想。

本文主要进行以下几个方面的研究:

第一部分为变系数 Sasa-Satsuma 方程的解析研究及怪波传播模拟。
本部分所研究的 Sasa-Satsuma 方程的变系数同时含有 $x$ 和 $t$。在对 Sasa-Satsuma 方程的解析研究中,本部分首先对方程的可积性进行了研究,求得了方程的 Painlev\'{e} 可积条件,并且在其中一组条件下扩展了 AKNS 系统求方程的 Lax 对。基于所求得的Lax对,得到了方程的自 B\"{a}cklund变换,结合种子解进一步求得方程的单孤子解。然后引进 Riccati 变量推导方程的无穷守恒律。最后,通过 Darboux 变换求得方程的一阶怪波对,并用 Mathematica 进行了模拟与分析。

第二部分为 6 阶常系数 KdV 方程的解析研究。
本部分研究的 KdV 方程最高次数是 6 阶,这使得其在求解上较为复杂,但是其解可描述更多的非线性现象。首先,通过贝尔多项式方法研究方程,得到贝尔多项式形式的双线性形式,根据双线性形式,求得方程的 $N$ 孤子解。同时结合贝尔多项式形式的双线性形式可以得到方程的双线性 B\"{a}cklund 变换,然后基于该变换对方程的可积性进行进一步分析,得到方程的 Lax 对,最后根据所求得的 Lax 对,分析推导,归纳出满足该方程的无穷守恒律。

最后一部分为自动求解方程怪波解程序包的开发。本部分主要借助符号计算工具 Mathematica 开发一个可以自动求解方程怪波解的程序包,程序包通过输入方程的双线性形式可以自动求解出方程的怪波解,减少了手工计算的复杂度,保证了计算的正确性,提高了研究效率。

\section{研究意义}
随着计算机科学技术的日益发展,符号演算系统逐渐成为人们关注的热点,如何应用符号演算工具辅助更多的科学领域发展具有重要的意义。非线性偏微分发展方程作为数学、物理、计算机等众多交叉学科共同涉及到的非线性问题,一直是学术界的研究的重点和难点,尤其是对非线性发展方程的解析研究更是具有重要意义。

非线性发展方程研究的最大困难之一在于解析解的获得,包括近年来迅速发展的怪波解的研究,大量经典书籍以及孤子文献都致力于获得更多新的解析解的方法介绍和学术研究,解析研究将直接影响非线性发展方程的理论和应用研究。

如何更好的利用符号演算系统Mathematica,对非线性发展方程的可积性以及解析解进行研究是本文研究的重点。随着对非线性发展方程深入地研究,所研究方程的复杂程度逐渐提高,因而必须借助符号计算系统。利用 Mathematica 对非线性发展方程的解析性质进行研究,对怪波进行模拟,进而将怪波求解过程进行符号化,编写出Mathematica的程序包是本文的主要研究内容,本文的研究成果对非线性发展方程的研究会起到一定的促进作用。

\section{论文结构安排}
本文一共五章,其中一、二章主要介绍背景和理论,三、四、五章是主要的研究内容,具体有如下安排:

第一章,绪论。本章首先对符号计算、非线性发展方程和怪波进行了简单的概述,
进而对怪波以及符号演算的研究发展现状进行整理和概括,然后阐述了本文的研究内容与研究意义,最后就论文的研究内容和结构安排进行了简明扼要的阐述。

第二章,理论和方法。本章详细介绍了后续研究非线性偏微分方程所使用的数学方法,其中所使用的数学
方法包括Painlev\'{e}分析、Lax可积、B\"{a}cklund变换、Hirota 双线性、贝尔多项式方法、无穷守恒律和怪波求解。

第三章,变系数 Sasa-Satsuma 方程。本章对变系数 Sasa-Satsuma 方程进行了解析研究、怪波求解、怪波模拟与分析。首先,对 Sasa-Satsuma 方程的研究背景及现状进行了阐述,通过 Painlev\'{e} 检测得到了方程的两组 Painlev\'{e} 可积条件,选择其中一组可积条件,根据变系数 Sasa-Satsuma 方程的形式扩展 AKNS 形式求得了 3$\times$ 3 阶 Lax 对,基于得到的 Lax 对求得 $\Gamma$-Riccati 形式的 B\"{a}cklund 变换,进而通过种子解得到了变系数 Sasa-Satsuma 方程的单孤子解,另外通过 Lax 对归纳推导出满足变系数 Sasa-Satsuma 方程的无穷多个守恒律,最后用 Darboux 变换方法求到了变系数 Sasa-Satsuma 方程的怪波解,并用 Mathematica 对怪波解进行了模拟与分析,详细阐明了变系数 Sasa-Satsuma 方程的各个变系数对方程怪波的波形及传播过程产生的影响。

第四章,6 阶 KdV 方程的解析研究。本章对 6 阶 KdV 方程进行解析研究。首先,通过贝尔多项式方法推导出 6 阶 KdV 方程的双线性形式,基于贝尔多项式形式的双线性形式求得方程的 N 孤子解、B\"{a}cklund 变换和 Lax 对,最后通过 Lax 对推导出方程的无穷守恒律。

第五章,怪波求解程序包开发。
本章主要开发求解方程怪波解的程序包,借助符号计算工具 Mathematica 开发一个可以自动求解方程怪波解的程序包,该程序包采用的算法不是直接求解方程怪波解,而是通过変量変换法寻找两个方程的关系,再将得到的关系式应用于其中一个方程的已知怪波解就可以得到另一个方程的怪波解。封装程序包减少了手工计算的复杂度,保证了计算的正确性并提高了研究效率。











